\documentclass[14pt]{extarticle}
\usepackage{amsmath}
\usepackage{float}
\usepackage{listings}
\lstset{
    %numbers=left,
    breaklines=true,
    tabsize=2,
    basicstyle=\ttfamily,
    literate={\ \ }{{\ }}1
}

\usepackage{graphicx}
\usepackage[T1]{fontenc}
\usepackage{hyperref}
\usepackage[utf8]{inputenc}
\usepackage{geometry}
 \geometry{
 a4paper,
 total={170mm,257mm},
 left=20mm,
 top=20mm,
 }

\usepackage{changepage}

%Legend
%\vspace{10pt} interlinea nome section - testo o Titolo elenco - descrizione elenco (Variable) 
\def\sp{\vspace{5pt}}
%\vspace{25pt} interlinea tra sections (Variable)
\def\ss{\vspace{25pt}}
%\vspace{3pt} interlinea tra linee in un paragrafo + breakline (Variable)
\def\pp{\vspace{10pt}\newline}
\def\ppn{\vspace{10pt}}
%Tab 
\def\tab{\hspace*{15pt}}

\usepackage{CJKutf8}
\newcommand{\zh}[1]{\begin{CJK}{UTF8}{gbsn}#1\end{CJK}}

%debug 1 -> debug, 0 -> finale
\newcounter{debug}
\setcounter{debug}{1}

\begin{document}
\title{title}
\author{author}
\date{date}

\begin{titlepage}
	\begin{figure}[t]
    		\centering\includegraphics[width=0.45\textwidth]{./Image/tongji-university1.png}
    		\centering\includegraphics[width=0.45\textwidth]{./Image/polito_logo_2021_blu_resized.png}
		\vspace{10mm}
	\end{figure}

	\begin{center}
	    	\textbf{ \LARGE{Tongji University\\}}
		%\textnormal{ \Large{Dipartimento di Automatica e Informatica Collegio di Ingegneria Informatica, del Cinema e Meccatronica\\}}
		\ss
		\textnormal{ \Large {MAURIZIO VASSALLO\\}}
		\textnormal{ \large {Student ID: \\}}
		\textnormal{ \large {Academic Tutor: a\\}}
		%\vspace{30mm}
	    		\vspace{\fill}\textbf{\LARGE{A Collision Avoidance System Using
Reinforcement Learning\\}}\vspace*{\fill}%vertical align
	\end{center}
	
\end{titlepage}

\tableofcontents
\newpage

\begin{center}
	\section{Introduction}
	\sp
\end{center}
\begin{flushleft}
	This report is drawn up after the development of the thesis project at Tongji University (\textbf{\zh{同济大学}}) for the Double Degree project Politong.
	\pp
	The thesis project aims to develop a machine learning algorithm that allows an autonomous vehicle to avoid obstacles. \\
	The research will focus on Reinforcement Learning methods.

\ppn
Nowadays it is possible to hear more and more about Autonomous Driving Vehicles. According to research firm, autonomous vehicles will match or exceed human safety by late 2020s and fulfill all mobility needs in 2040s to 2060s. Optimists predict that by 2030, autonomous vehicles will be sufficiently reliable, affordable and 
common to displace most human driving, providing huge savings and benefits. However, there are good reasons to be skeptical. There is considerable uncertainty concerning autonomous vehicle development, benefits and costs, travel impacts, and consumer demand. Considerable progress is needed before autonomous vehicles can operate reliably in mixed urban traffic, heavy rain and snow, unpaved and unmapped roads, and where wireless access is unreliable\cite{AVfirm}.

\ppn
The autonomous cars(also known as a self-driving cars or a driverless cars) are a vehicle that are capable of sensing its environment and navigating without human input
The ability of autonomous vehicles to operate without human intervention depends
on their level of technological sophistication, in accordance with the current six-degree
autonomy scale proposed by the International Society of Automotive Engineers (\textbf{SAE})\cite{AVtaxonomy}.
There are 6 levels of driving automation, from level 0 (no automation) to level 5 (full unlimited automation); intermidiate levels (1 to 3) are considered semi-autonomous\cite{AVlevels,AVlevels2}. \\
Currently we are between level 2 and 3, so even if the current technology is behind the famous Level 5 of driving automation, there is lot of work to make it happen.

\ppn
The idea behind these cars is quite simple: outfit the vehicles with sensors that can track all the objects around and make the cars understand the world around them. Autonomous vehicles are driven using technology such as GPS, odometry, radars, laser lights and other devices\cite{AVlevels2}.
 These sensors themselves do not make the car ‘smart’, what make it autonomous are the big computers inside and the algorithms they are running. Usually these softwares run neural networks: these take as input the sensor recordings, elaborate them and output some values like steering angle, accelerate, brake or other important values.
 Even if the idea behind these technological innovative vehicles is simple the implementation is not: not enough hardware computation, not enough training data, problems with handle different weather conditions(fog, rain, snow, etc.),  the current regolation remains in a nascent stage.
 \pp
 Even if this tecnology is not diffused yet, there are many potential benefits that autonomous vehicles could introduce in our society:
 \begin{itemize}
 \item \textbf{Transportation Safety}. The most notable predicted benefit of autonomous vehicle technology is a substantial reduction in the human and economic toll of traffic accidents. ndeed,  impairment,  distractions,  and  fatigue  alone  account  for over 50\% of all fatal crashes. The use of autonomous vehicles could significantly reduce the incidence of such crashes, as vehicles with no human operators  are  never  drunk,  distracted,  fatigued,  or  otherwise  susceptible  to human failings.
\item \textbf{Access to Transportation}. Another important potential benefit of autonomous vehicle technology  is  increased  mobility  for  populations  currently  unable  or  not 
permitted  to  operate  traditional  vehicles. These  populations  include  older 
citizens, the disabled, people too young to drive, and others without a driver’s 
license.
\item \textbf{Traffic Congestion and Land Use}. Autonomous vehicles could reduce congestion and change the way in which cities are planned. Most car are moving only for 5\% of their lives, for the 95\% they are parked\cite{AVparking}! For this reason lot of space is dedicated to parking lots; the same space that could be used for different purposes (green spaces, etc.).
\item \textbf{Energy and Emissions}. Autonomous vehicle technology has the potential to reduce both energy consumption and pollution thanks to efficiencies gained through smoother acceleration and deceleration and  increased  roadway capacity.\cite{AVbenefit}
 \end{itemize}

	\ss
\end{flushleft}

\newpage
\begin{center}
	\section{The}
	\sp
\end{center}
\begin{flushleft}
ience and in the back-end server development part. 
	\ss
\end{flushleft}

\subsection{Test}
\sp
\begin{flushleft}
The first task was .
\end{flushleft}

\subsection{Complex-Analysis} \label{complex}
\sp
\begin{flushleft}
This part concerns the st
\subsubsection{Dataset }
m movements
\end{flushleft}



\newpage
\begin{center}
	\section{Conclusion}
	\sp
\end{center}
\begin{flushleft}

This internship was a formative experience that allowed me to discover many new topics but also to understand how a company works. Moreover, it allowed me to understand that there is a substantial difference between the university environment and the corporate environment: university, as mentioned by some professors during the courses, has the task of giving a general knowledge of the topics and the real task is to give a way of thinking and analysing a problem to find a solution; indeed, during this period in the company I realized that computer knowledge was not always essential but it was more important to understand what was needed to be done and how to do it.
\ppn

I believe that this internship was very useful because it allowed me to deepen and learn about new topics and gave me an idea to understand if the tasks assigned could be addressed in a future university career or in a possible job.
\newline
This is also an experience that allows to grow both personally and professionally, not to mention that it is a way to
enrich the Curriculum Vitae.

\ppn
I am fully satisfied with the internship as AI researcher and developer and I am satisfied with the atmosphere of serenity and professionalism that I lived in the company thanks to the kind and extremely helpful colleagues.
	\ss
\end{flushleft}

%References
\newpage
\begin{thebibliography}{999}

\bibitem{AVfirm}
  Todd Litman. 2021.
  \emph{Autonomous Vehicle Implementation Predictions}.\\
  \url{https://www.vtpi.org/avip.pdf}
  
   \bibitem{AVtaxonomy}
  SAE. 2016.
  \emph{Taxonomy and Definitions for Terms Related to Driving Automation Systems for On-Road Motor Vehicles}.\\
  \url{https://www.sae.org/standards/content/j3016_201806/}
   
  \bibitem{AVlevels}
  Monika Stoma, Agnieszka Dudziak, Jacek Caban and Paweł Drozdzie. 2021.
  \emph{The Future of Autonomous Vehicles in the Opinion of Automotive
Market Users}.\\
  \url{https://www.mdpi.com/1996-1073/14/16/4777/pdf}
   
   \bibitem{AVlevels2}
  Piotr CZECH, Katarzyna TUROŃ, Jacek BARCIK. 2018.
  \emph{AUTONOMOUS VEHICLES: BASIC ISSUES}.\\
  \url{https://doi.org/10.20858/sjsutst.2018.100.2}
  
     \bibitem{AVparking}
  Paul Barter. 2013.
  \emph{"Cars are parked 95\% of the time". Let's check!}.\\
  \url{https://www.reinventingparking.org/2013/02/cars-are-parked-95-of-time-lets-check.html}
  
   \bibitem{AVbenefit}
  Jeremy A. Carp. 2018.
  \emph{AUTONOMOUS VEHICLES: PROBLEMS AND PRINCIPLES FOR FUTURE REGULATION }.\\
  \url{ https://scholarship.law.upenn.edu/cgi/viewcontent.cgi?article=1048&context=jlpa}
 
   

\end{thebibliography}

\end{document}
